\chapter{Final conclusions and future prospects}

\noindent For radio-based ultra-high-energy neutrino searches to succeed, possible backgrounds have to be carefully considered and understood. The radio spectrum is populated by telecommunication signals in every direction, power and frequency. Normally, these signals are easily filtrated if they are constant and well-known, but this is not always the case. \\

\noindent In the case of horizon-based signals, weather-induced variability complicates the modeling and reconstruction of signals directions. This is critical because next-generation astroparticle telescopes will target the horizon to maximize their effective volume for neutrino detection. Therefore, determining whether atmospheric effects can create unexpected detector signatures is of great importance. \\

\noindent The main focus of this thesis is evaluating the effects of a specific atmospheric propagation scenario: hydrometeor scatter, commonly associated to poor weather. Due to its transient and omnidirectional nature, the resulting signals are more difficult to distinguish from astroparticle events than fair weather phenomena. This investigation is motivated by recent evidence that atmospheric perturbation (clouds and rain) might be redirecting distant anthropogenic RFI towards detectors.\\

\noindent The approach chosen for this work is simulation-based. The computational code is a preexisting software in the field, which was adapted for this purpose. The core simulation involves a transmitter illuminating a designated volume filled with hydrometeors. These are modeled as dense particles within the Rayleigh scattering regime to simplify the physics. The code computes the scattered electromagnetic waves and, at a simulated receiver, derives key parameters like the radar cross section, received electric field and voltage. \\

\noindent As a first study of its kind, this work adopts a conceptual approach to isolate the core physical phenomenon. The simulations use a controlled, directional beam, similar to an antenna array's beamforming, to probe a specific atmospheric region. This concentrates power on a target, providing a clear measure of scattering efficiency. A full parameterization, including a detailed antenna response model, is the necessary next step, for which this work establishes the essential framework and proof of concept. \\

\noindent The received voltage was evaluated against an arbitrary but typical detection threshold of 1 $\mu V$ to identify the parameter space configurations that produce a non-negligible hydrometeor scatter signal. For quasi-forward-scatter geometries (scattering angles of $\theta = 7^{\circ} - 14^{\circ}$, the geometry corresponding to near-horizon detectors), the received voltage exceeds the threshold for total path lengths $R_{\text{T}}+R_{\text{R}}$ of up to 500 km, provided the effective transmitted power is at least 1 kW. In contrast, a more demanding geometry ($\theta=90^{\circ}$), raises the power requirement to 100 kW to produce a comparable signal in the same range. \\

\noindent In the domain of the clear-sky effects, powerful phenomena that could be evaluated in the future are are ducting, which has the potential of enhancing the transmitted signal, and reflection by the melting layer, which is theoretically more difficult to model because of the complexity of mixed-phase hydrometeors. Future work should also investigate the effects due to the ionosphere, especially sporadic E events which involve the transient presence of thin plasma capable of reflecting certain radio frequencies. \\

\noindent A valuable source to address atmospheric effects is the COST 210 \cite{ballabioCOST210Influence1991}, a report from the Commission of the European Communities. This resource provides valuable models for predicting how the atmosphere bends and focuses radio signals, creating anomalous propagation that can enhance distant terrestrial transmissions. Originally developed for telecom engineering applications, its procedures are directly applicable to radio astronomy for identifying site-specific meteorological conditions that increase vulnerability to RFI. By detailing how atmospheric effects alter a signal's angle of arrival and fading, the report enables the development of robust algorithms to veto or filter data during periods of high anomalous propagation risk. \\

\noindent This work demonstrates that reflected continuous waves (CW) from overcast skies could be detected. This poses a concern for near-horizon neutrino searches. In fact, bandwidth limitations of many radiotelescopes complicate the separation between low CW signals and astrophysical impulsive signals. Moreover, anthropogenic impulsive signals with a similar power budget are expected to represent a comparable threat too. Based on the findings presented, hydrometeor-scattered RFI cannot be ruled out as a possible source of background in radio experiments. \\

\noindent Should radio reflections from clouds be confirmed as a measurable background, near-horizon observatories will require challenging new mitigation strategies. RFI can either be non-impulsive, which includes continuous or periodic emissions from sources like oscillators and switching power supplies; or impulsive, characterized by broadband transient signals due to power line arcing and industrial sparking. The latter being distinguishable from neutrino signals only by the initial polarization and direction. Unfortunately, atmospheric effects, first and foremost hydrometeor scatter, manipulate these traits. Consequently, a likely approach is to perform a sophisticated statistical analysis that incorporates real-time meteorological data. A definitive neutrino detection from near-horizon observatories will be complex, requiring the identification of a small, isotropic excess of events against a dynamic RFI background, a process that carries significant systematic uncertainties.